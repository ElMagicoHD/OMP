\documentclass[a4paper, 12pt]{article}

\usepackage[T1]{fontenc}
\usepackage[ngerman]{babel}
\usepackage[utf8]{inputenc}
\usepackage{hyperref}
\usepackage{setspace}
\usepackage{todonotes}
\usepackage{graphicx}
\usepackage{float}
\usepackage[backend=biber, style=ieee, bibencoding=inputenc, sortcites=false]{biblatex}
\usepackage[autostyle, german=quotes]{csquotes}
\DeclareFieldFormat{urldate}{\mkbibbrackets{\bibstring{urlseen}\space#1}}

\onehalfspacing

\title{\huge Implementierung und Evaluation eines effizienten Algorithmus für den optimalen Treffpunkt in gewichteten Graphen}

\date{\today}
\author{}

\bibliography{resources}
\begin{document}

\maketitle
\vspace*{\fill}
\begin{flushleft}
\begin{tabular}{l}
Bachelorarbeit von Alexander Linger \\
Matrikelnummer: 396673 \\
\\
Technische Universität Berlin \\
B.Sc. Informatik
\\
\\
Abgabetermin: TBD
\end{tabular}
\end{flushleft}
\newpage
\section{Motivation}
\subsection{Wofür?}
Das effiziente berechnen des optimalen Treffpunktes (OMP) kann in diversen Situationen nützlich sein. Denkbar ist zum einen das Bestimmen eines Treffpunktes für eine Gruppe, beispielsweise für Reiseunternehmen um einen Abholstandort mit minimalem Gesamtweg für die Gruppe zu ermitteln \cite[vgl.][S. 277]{Tiwari_Kaushik_2013}. Ein weiteres Einsatzszenario sind Versanddienstleistungsunternehmen: Zwei Transporter von zwei verschiedenen Logistikzentren, welche sich treffen um Ware untereinander auszutauschen um somit Kosten zu reduzieren \cite[vgl.][S.1 f.]{huang_meet_2018}. Ein Szenario welches auch in dieser Arbeit simuliert wird, ist das bestimmen des OMPs für Roboter. Aufgaben bei denen Roboter kollaborativ zusammen arbeiten, erfordern meist, dass sich Roboter treffen, bevor sie deren Tätigkeit ausüben \cite[vgl.][S. 107]{Lanthier_Nussbaum_Wang_2005}.
\end{document}
